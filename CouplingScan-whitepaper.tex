\documentclass[a4paper, 11pt,notoc]{article}
\pdfoutput=1
\usepackage{jcappub}
\usepackage{graphicx}
\usepackage{booktabs}
\usepackage{verbatim}
\usepackage{caption}
\usepackage{xspace}
\usepackage{hyperref}
\usepackage{multirow}
\usepackage{placeins}
\usepackage{array}
\usepackage{subcaption}

% Variables
\newcommand{\sigv}{\ensuremath{\langle \sigma v_{\rm{rel}} \rangle}\xspace}
\newcommand{\MET}{\ensuremath{E_T^\mathrm{miss}}\xspace}
\newcommand{\met}{\MET}
\newcommand{\MT}{\ensuremath{M_{T}}\xspace}
\newcommand{\pt}{\ensuremath{p_{T}}\xspace}

% Units
\newcommand{\GeV}{\textrm{GeV}\xspace}
\newcommand{\gev}{\GeV\xspace}
\newcommand{\TeV}{\textrm{TeV}\xspace}
\newcommand{\tev}{\TeV\xspace}

% Particle names
\newcommand{\lp}{\ensuremath{l^{+}}\xspace}
\newcommand{\lm}{\ensuremath{l^{-}}\xspace}
\newcommand{\ttbar}{\ensuremath{\bar{t}t}}
\newcommand{\bbbar}{\ensuremath{\bar{b}b}}
\newcommand{\A}{A}
\newcommand{\pa}{a}
\newcommand{\pH}{H}
\newcommand{\pZ}{Z}
\newcommand{\Hc}{\ensuremath{H^{\pm}}\xspace}

% Particle masses
\newcommand{\mDM}{\ensuremath{M_{\chi}}\xspace}
\newcommand{\mdm}{\ensuremath{M_{\chi}}\xspace}
\newcommand{\mmed}{\ensuremath{M_{\rm{med}}}\xspace}
\newcommand{\mMed}{\ensuremath{M_{\rm{med}}}\xspace}
\newcommand{\mZ}{\ensuremath{M_{\rm{Z}}}\xspace}
\newcommand{\mA}{\ensuremath{M_{A}}\xspace}
\newcommand{\ma}{\ensuremath{M_{a}}\xspace}
\newcommand{\mH}{\ensuremath{M_{H}}\xspace}
\newcommand{\mHc}{\ensuremath{M_{H^{\pm}}}\xspace}
\newcommand{\mh}{\ensuremath{M_{h}}\xspace}
\newcommand{\mt}{\ensuremath{M_{t}}\xspace}

% Couplings
\newcommand{\gDM}{\ensuremath{g_{\rm{DM}}}\xspace}
\newcommand{\gq}{\ensuremath{g_q}\xspace}
\newcommand{\gSM}{\gq}
\newcommand{\gdm}{\gDM}
\newcommand{\ifb}{\ensuremath{\rm{fb}^{-1}}\xspace}

% Other parameters
\newcommand{\sinp}{\ensuremath{\sin\theta}\xspace}
\newcommand{\cosp}{\ensuremath{\cos\theta}\xspace}
\newcommand{\sinbma}{\ensuremath{\sin(\beta - \alpha)}\xspace}
\newcommand{\cosbma}{\ensuremath{\cos(\beta - \alpha)}\xspace}
\newcommand{\tanb}{\ensuremath{\tan\beta}\xspace}
\newcommand{\lap}[1]{\lambda_{P#1}} % can use like \lap1 , \lap2
\newcommand{\lam}[1]{\lambda_{#1}} % can use like \lam3

% Search channels
\newcommand{\hdm}{\ensuremath{h+\textrm{DM}}\xspace}
\newcommand{\monoh}{\ensuremath{h+\MET}\xspace}
\newcommand{\monohbb}{\ensuremath{h(bb)+\MET}\xspace}
\newcommand{\monoz}{\ensuremath{Z+\MET}\xspace}
\newcommand{\monozll}{\ensuremath{Z(\ell\ell)+\MET}\xspace}
\newcommand{\monozhad}{\ensuremath{Z(\textrm{had})+\MET}\xspace}

% Software/Program/Model names
\newcommand{\mg}{\textsc{MadGraph~5}\xspace}
\newcommand{\mgamcnlo}{MG5\_aMC@NLO\xspace}
\newcommand{\dmsimp}{\textsc{DMsimp}\xspace}
\newcommand{\maddm}{\textsc{MadDM}\xspace}
\newcommand{\hdma}{\ensuremath{\textrm{2HDM+a}}\xspace}

% Misc
\newcommand{\GamA}{\ensuremath{\Gamma_{A}}\xspace}
\newcommand{\Uli}{\color{red}}
\definecolor{cerulean}{RGB}{44,150,207}
\newcommand{\ATLASComments}{\color{cerulean}}
\newcommand{\bra}[1]{\langle #1|}
\newcommand{\ket}[1]{|#1\rangle}
\newcommand{\sens}{\mathcal{S}\xspace}
\newcommand{\senstot}{\mathcal{S}_\textrm{tot}\xspace}
\newcommand{\mathsc}[1]{\text{\textsc{#1}}}
\newcommand{\vev}[1]{\langle {#1} \rangle}

\DeclareMathOperator{\arccot}{arccot}

\def\be   {\begin{equation}}   \def\ee   {\end{equation}}
\def\ba   {\begin{array}}      \def\ea   {\end{array}}
\def\bea  {\begin{eqnarray}}   \def\eea  {\end{eqnarray}}
\def\bean {\begin{eqnarray*}}  \def\eean {\end{eqnarray*}}
\def\nn{\nonumber}


\allowdisplaybreaks

%DIF PREAMBLE EXTENSION ADDED BY LATEXDIFF
%DIF UNDERLINE PREAMBLE %DIF PREAMBLE
\RequirePackage[normalem]{ulem} %DIF PREAMBLE
\RequirePackage{color}\definecolor{RED}{rgb}{1,0,0}\definecolor{BLUE}{rgb}{0,0,1} %DIF PREAMBLE
\providecommand{\DIFadd}[1]{{\protect\color{blue}\uwave{#1}}} %DIF PREAMBLE
\providecommand{\DIFdel}[1]{{\protect\color{red}\sout{#1}}}                      %DIF PREAMBLE
%DIF SAFE PREAMBLE %DIF PREAMBLE
\providecommand{\DIFaddbegin}{} %DIF PREAMBLE
\providecommand{\DIFaddend}{} %DIF PREAMBLE
\providecommand{\DIFdelbegin}{} %DIF PREAMBLE
\providecommand{\DIFdelend}{} %DIF PREAMBLE
%DIF FLOATSAFE PREAMBLE %DIF PREAMBLE
\providecommand{\DIFaddFL}[1]{\DIFadd{#1}} %DIF PREAMBLE
\providecommand{\DIFdelFL}[1]{\DIFdel{#1}} %DIF PREAMBLE
\providecommand{\DIFaddbeginFL}{} %DIF PREAMBLE
\providecommand{\DIFaddendFL}{} %DIF PREAMBLE
\providecommand{\DIFdelbeginFL}{} %DIF PREAMBLE
\providecommand{\DIFdelendFL}{} %DIF PREAMBLE
%DIF END PREAMBLE EXTENSION ADDED BY LATEXDIFF

\def\bm#1{\mbox{\boldmath$#1$\unboldmath}} 

\begin{document}
\title{\begin{boldmath} \huge LHC Dark Matter Working Group:  \\ Title of whitepaper \vspace{7mm} \end{boldmath}}

%%%%%%

%Add your name here!

\author[4]{Andreas~Albert,}
\affiliation[4]{III. Physikalisches Institut A, RWTH Aachen University, \\
Physikzentrum, Otto-Blumenthal-Stra{\ss}e, Aachen, Germany}

\author[10,*]{Antonio~Boveia,}
\affiliation[10]{Ohio State University and Center for Cosmology and Astroparticle Physics, \\
191 W. Woodruff Avenue Columbus, OH 43210, USA}

\author[11]{Oleg~Brandt,}
\affiliation[11]{Kirchhoff-Institut f{\"u}r Physik, Ruprecht-Karls-Universit{\"a}t Heidelberg, \\ 
Im Neuenheimer Feld 227, 69120 Heidelberg, Germany}

\author[12,*]{Caterina~Doglioni,}
\affiliation[12]{Fysiska institutionen, Lunds universitet, Professorsgatan 1, Lund, Sweden}

\author[21,22,23,*]{Ulrich~Haisch,}
\affiliation[21]{Max Planck Institute for Physics, F{\"o}hringer Ring 6,  80805 M{\"u}nchen, Germany}
\affiliation[22]{Rudolf Peierls Centre for Theoretical Physics, University of Oxford, \\ 
Oxford, OX1 3PN, UK}
\affiliation[23]{Theoretical Physics Department, CERN, CH-1211 Geneva 23, Switzerland}

%\author[28]{Greg~Landsberg,}
%\affiliation[28]{Brown University, Dept. of Physics, 182 Hope St, Providence, RI 02912, USA}
\author[28]{David~Yu,}
\affiliation[28]{Brown University, Dept. of Physics, 182 Hope St, Providence, RI 02912, USA}


\author[24]{Katherine~Pachal,}
\affiliation[24]{Duke University, Durham, NC 27708, USA}

\author[8,35]{Priscilla~Pani,}
\affiliation[35]{DESY Zeuthen, Platanenallee 6, 15738 Zeuthen, Germany}

\author[17,*]{Tim~M.~P.~Tait,}

\affiliation[*]{DMWG organisers}

\hfill CERN-LPCC-2018-XX

\abstract{
Dark matter is one of the main science drivers of the particle and astroparticle physics community. Pursuing the understanding of the nature of dark matter requires a broad approach, with multiple experiments pursuing different experimental hypotheses.

Dark matter particles could be produced at particle colliders. Dark matter searches at collider experiments provide insight on dark matter complementary to searches in direct/indirect detection and experiments, and to astrophysical evidence.

In order to compare results from a wide variety of experiments, a common theoretical framework is required. Among the numerous theoretical frameworks that describe dark matter (see e.g.~\cite{doi:10.1146/annurev-nucl-101917-021008}, and \cite{Kahlhoefer:2017dnp,doi:10.1146/annurev-astro-082708-101659} for collider-focused reviews), the ATLAS and CMS experiments at the Large Hadron Collider have adopted a series of simplified models that include dark matter particles as benchmarks for their searches~\cite{ABERCROMBIE2020100371}.

In these models, the interaction between Standard Model (SM) and dark matter particles is mediated by a new particle, called a mediator. The interaction strength is controlled by the couplings of the mediator to dark matter and to SM particles. 

So far, the presentation of LHC results (as well as the presentation of projections of future hadron collider experiments) has focused on four benchmark scenarios with different choices of couplings to quarks and leptons, as recommended by the Dark Matter Working Group~\cite{BOVEIA2020100365, ALBERT2019100377}.

In this work, we plan to describe methods to extend those four benchmark scenarios to scenarios with arbitrary couplings, and release the corresponding code for use in further studies and projections of collider dark matter searches in the framework of simplified models. This will extend the applicability of the comparisons of collider searches to accelerator experiments that are sensitive to smaller couplings, and give a more complete picture of the coupling dependence of the sensitivity of dark matter collider searches when compared to direct and indirect detection searches. 

We plan to focus on s-channel DM simplified models, where the mediator particle has vector, axial-vector, scalar and pseudoscalar couplings with DM and SM particles. This work will cover collider searches for visible decays of the mediator particles, as well as for searches targeting the invisible particles via the associated production of one or more SM particles~\cite{ATL-PHYS-PUB-2020-021,CMSSummary}.
}  

\maketitle

\newpage 

%%%%%%%%%%%%%%%%%%%%%%%%%%%%%%%%%%%%%%%%%%%%%%%%%%%%%%%%%%%%%%%%%%%%
%%%%%%%%%%%%%%%%%%%%%%%%%%%%%%%%%%%%%%%%%%%%%%%%%%%%%%%%%%%%%%%%%%%%
%%%%%%%%%%%%%%%%%%%%%%%%%%%%%%%%%%%%%%%%%%%%%%%%%%%%%%%%%%%%%%%%%%%%

\section{Introduction}
\label{sec:introduction}

Explain goal - to be able to scale results in one model/set of parameters to any other with reasonable accuracy without resorting to generating events.


%%%%%%%%%%%%%%%%%%%%%%%%%%%%%%%%%%%%%%%%%%%%%%%%%%%%%%%%%%%%%%%%%%%%
\section{Models considered}
\label{sec:models}

Considering both vector and axial-vector mediator $s$-channel models - be sure to add citations to original DMWG papers.


%%%%%%%%%%%%%%%%%%%%%%%%%%%%%%%%%%%%%%%%%%%%%%%%%%%%%%%%%%%%%%%%%%%%
\section{Resonant final states}


\subsection{Di-jet and di-jet+X}

\subsection{Di-lepton}

Discuss the di-lepton final state requires accounting for widths more carefully since the resolution is much better.


%%%%%%%%%%%%%%%%%%%%%%%%%%%%%%%%%%%%%%%%%%%%%%%%%%%%%%%%%%%%%%%%%%%%
\section{Mono-x final states}



%%%%%%%%%%%%%%%%%%%%%%%%%%%%%%%%%%%%%%%%%%%%%%%%%%%%%%%%%%%%%%%%%%%%
\section{Example}

Use existing ATLAS or CMS public data from HEPData to show this working?


%%%%%%%%%%%%%%%%%%%%%%%%%%%%%%%%%%%%%%%%%%%%%%%%%%%%%%%%%%%%%%%%%%%%
\section{Conclusion}


%%%%%%%%%%%%%%%%%%%%%%%%%%%%%%%%%%%%%%%%%%%%%%%%%%%%%%%%%%%%%%%%%%%%
%%%%%%%%%%%%%%%%%%%%%%%%%%%%%%%%%%%%%%%%%%%%%%%%%%%%%%%%%%%%%%%%%%%%
%%%%%%%%%%%%%%%%%%%%%%%%%%%%%%%%%%%%%%%%%%%%%%%%%%%%%%%%%%%%%%%%%%%%

\acknowledgments 

[To be updated] The research of A.~Boveia is supported by the U.S. DOE grant  DE-SC0011726. C.~Doglioni has received funding from the European Research Council under the European Union's Horizon 2020 research and innovation program (grant agreement 679305) and from the Swedish Research Council. U.~Haisch acknowledges the hospitality and support of the CERN Theoretical Physics Department. The work of T.~M.~P.~Tait is supported in part by NSF grant PHY-1316792. We gratefully acknowledge the support by the U.S. DOE. 

%%%%%%%%%%%%%%%%%%%%%%%%%%%%%%%%%%%%%%%%%%%%%%%%%%%%%%%%%%%%%%%%%%%
%%%%%%%%%%%%%%%%%%%%%%%%%%%%%%%%%%%%%%%%%%%%%%%%%%%%%%%%%%%%%%%%%%%%
%%%%%%%%%%%%%%%%%%%%%%%%%%%%%%%%%%%%%%%%%%%%%%%%%%%%%%%%%%%%%%%%%%%%

\appendix

\section{Appendix}
\label{app:recast}

Document public code here?

\newpage 

\bibliography{CouplingScan-whitepaper}
\bibliographystyle{JHEP}



\end{document}
